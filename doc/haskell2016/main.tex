\documentclass{beamer}
%
% Choose how your presentation looks.
%
% For more themes, color themes and font themes, see:
% http://deic.uab.es/~iblanes/beamer_gallery/index_by_theme.html
%
\mode<presentation>
{
  \usetheme{default}      % or try Darmstadt, Madrid, Warsaw, ...
  \usecolortheme{default} % or try albatross, beaver, crane, ...
  \usefonttheme{default}  % or try serif, structurebold, ...
  \setbeamertemplate{navigation symbols}{}
  \setbeamertemplate{caption}[numbered]
} 

\usepackage{url}
\usepackage[english]{babel}
\usepackage[utf8x]{inputenc}
\usepackage{color}
\usepackage{listings}
\usepackage{amssymb,amsmath,amsthm}
\usepackage{tikz}
\usepackage{mathpartir}
\usepackage{fancyvrb}
\usetikzlibrary{shapes,arrows}

\newcommand{\tname}{\textsc{Autobahn}}
\newcommand{\dppsi}{\textsc{PSI}}
\newcommand{\cut}[1]{}
\newcommand{\stepsto}[2]{\ensuremath{#1 \Downarrow #2}}
\newcommand{\subst}[3]{\ensuremath{#1\{#3/#2\}}}
\newcommand{\trans}[4]{\ensuremath{\mathit{transform}(#1 : #2) :  #3 ~\{~#4~\}}}
\newcommand{\numinter}[2]{\ensuremath{N[#1,#2]}}
\newcommand{\wf}{\mbox{ wf}}
\newcommand{\br}[1]{\langle #1 \rangle}
\def\Yields{\Downarrow}
\newcommand{\tyex}{\sigma}
\newcommand{\typesto}[4]{{#1} \vdash {#2} : {#3} \vartriangleright {#4}}
\newcommand{\homepage}{\url{remysucre.github.io}}

\title[Title]{\tname{} : Using Genetic Algorithms to Infer Strictness Annotations}
\author{Yisu Remy Wang, Diogenes Nunez, Kathleen Fisher}
\institute{Tufts University, Medford MA, USA}
  
\date{\today}

\begin{document}

\begin{frame}
  \titlepage
\end{frame}

% Uncomment these lines for an automatically generated outline.
%\begin{frame}{Outline}
%  \tableofcontents
%\end{frame}

\section{Overview}

\begin{frame}{Overview}

\begin{itemize}
  \item Intro. 
  \item The Problem
  \item Background: (Strictness Annotations) \& Genetic Algorithms
  \item The Algorithm
  \item Soundness
  \item Evaluation \& Case Study
  \item Related Work
  
\end{itemize}

%\vskip 1cm
%
%\begin{block}{Examples}
%Some examples of commonly used commands and features are included, to help you get started.
%\end{block}

\end{frame}

%\section{Some \LaTeX{} Examples}
%
%\subsection{Tables and Figures}

\section{Intro}
\begin{frame}{Once Upon a Time, 2 Eager Programmers with a Lazy Program \dots\footnote{View animation in Acrobat Reader}}
\begin{itemize}
\item 1st slide: code on the left, pic / animation on the right showing thunk building up
\item 2nd slide: annotated code, animation showing absence of thunk
\end{itemize}
\end{frame}

\begin{frame}{The Problem \& Solutions}
\begin{itemize}
\item \textbf{The problem}: how much laziness?
  \begin{itemize}
    \item Too much laziness \textbf{slowdown programs}: thunk leak, generating ``cheap'' thunks
    \item Too little laziness \textbf{wastes computation}, contradicts the spirit of Haskell
    \item \textbf{Library writers} cannot know how much laziness is good
  \end{itemize}
\item \textbf{Solutions}: 
  \begin{itemize}
    \item ghc strictness analysis: must be \textbf{conservative} for soundness
    \item manual annotations: difficult to know \textbf{effectiveness} and \textbf{soundness}
  \end{itemize}
\item \textbf{\tname{}}:
  \begin{itemize}
    \item more aggressive in exploring strictness optimizations
    \item helps the programmer with in the task of annotating programs: find a 
          effective one
  \end{itemize}
\end{itemize}
\end{frame}

\begin{frame}{Background: Genetic Algorithms}
  \begin{itemize}
    \item Have a picture for GA representation \& execution
  \end{itemize}
\end{frame}

\begin{frame}{Background: GA for Strictness Annotations}
  \begin{itemize}
    \item Why is it good: avoid local optima
    \item Works like a desperate Haskeller trying all kinds of different bangs
    \item Works great if bangs work with each other in a simple way (corpus analysis will answer this)
    \item !! bangs can only have complex relation when they share some scope, investigate this!
  \end{itemize}
\end{frame}

\begin{frame}{Algorithm: Representation}
  \begin{itemize}
  \item Genes \& chromosomes
  \item Fitness Functions: the user has the opportunity to optimize over any 
                            metric Haskell RTS reports
  \end{itemize}
\end{frame}

\begin{frame}{The Algorithm: Optimization}
  \begin{itemize}
  \item Parameters
  \item 1st Generation
  \item New Generations
  \item Determining a Winner
  \end{itemize}
\end{frame}

\begin{frame}{The Algorithm: Pulling it All Together}
  \begin{itemize}
  \item List of inputs
  \item Start demo and leave running?
  \end{itemize}
\end{frame}

\cut{
\begin{frame}{The Algorithm: Discussion}
  \begin{itemize}
  \item seq, strict app etc. 
  \end{itemize}
\end{frame}
}

\begin{frame}{Soundness}
  \begin{itemize}
  \item use example for soundness problem
  \item requiring representitive input is good enough
  \end{itemize}
\end{frame}

\cut{
\begin{frame}{Evaluation}
  \begin{itemize}
  \item Introduce benchmarks / setup
  \end{itemize}
\end{frame}
}

\begin{frame}{Evaluation - Case Study: gcSimulator}
\end{frame}

\begin{frame}{Evaluation - Case Study: Aeson}
  \begin{itemize}
  \item Specializing libraries
  \end{itemize}
\end{frame}

\begin{frame}{Evaluation: strict Haskell}
  \begin{itemize}
  \item ?? Did we ever try StrictHaskell on aeson/gcSim?
  \end{itemize}
\end{frame}

\begin{frame}{Evaluation: 10-fold Cross-validation}
  \begin{itemize}
  \item Stability of optimization
  \end{itemize}
\end{frame}

\begin{frame}{Evaluation: Autobahn Performance}
  \begin{itemize}
  \item room for improvement: parallelization, use GHC-API
  \end{itemize}
\end{frame}

\begin{frame}{Related Work}
  \begin{itemize}
  \item static analysis
  \item including dynamic information
  \item other approaches
  \end{itemize}
\end{frame}

\begin{frame}{Future Work}
  \begin{itemize}
  \item Soundness
  \item Other algorithms
  \item Other annotations
  \item Improve the tool
  \item Learnabile?
  \item Learn to be annotation expert
  \end{itemize}
\end{frame}

\begin{frame}{Acknowledgments}
We thank ourselves for our hardwork. 
\end{frame}

\end{document}
